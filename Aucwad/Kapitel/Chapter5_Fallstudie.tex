\chapter{Fallstudie\label{Chapter5}}

In diesem Kapitel wird die Anwendung des Programms auf das vorliegende Test Objekt erl�utert. 

\section{Test System}

In allen Tests wie auch bei der Erstellung des Programms wurde folgendes System verwendet:

HP EliteBook 8740w

\begin{table}[htbp]
\begin{tabular}{|c|c|}
\hline 
CPU & Intel Core i7 M640 @ 2.80 Ghz \\ 
\hline 
RAM & 4,00 GB DDR3 \\ 
\hline 
Festplatte & Seagate Momentus 7200.4 250GB, SATA II (ST9250410AS) \\ 
\hline 
Betriebssystem & Windows 8, 64-Bit \\ 
\hline 
Java Version & 1.7.0\_10 \\ 
\hline 
Firefox Version & 17.0 \\ 
\hline 
\end{tabular}
\label{tab:data}
\end{table}

F�r das Programm werden nur Java-Versionen ab 1.7.0\_10 aufw�rts unterst�tzt. Als Browser wird nur der Mozilla Firefox in Version 17.0 oder geringer unterst�tzt, da zu Beginn der Programmierung die Selenium Bibliothek in der Version 2.25 eingebunden wurde, welche nur diese Versionen von Mozilla Firefox unterst�tzt.

\TODO{BSP aus benchmar.tex?}

\section{Test Objekt}

\subsection{Test Wizard}

\section{Testzeiten}

\TODO{Vorgehensweise, Beispiel}

Auf dem Testsystem wurden mehrere Tests ausgef�hrt um eine Testzeitanalyse zu erstellen. F�r 10 Tests, welche sich auf eine Test-Seite beziehen, ben�tigt das Programm im Schnitt 10.3 Sekunden. Hochgerechnet auf 13.000 Tests(\ref{sec:Szenario}) entspricht das in etwa 3.75 Stunden. 
Ein Test �ber zwei Seiten mit zwei Testscases f�r die erste Seite und f�nf f�r die zweite Seite dauert durchschnittlich 208 Sekunden. Das bedeutet das 10 Tests mit einem Seiten�bergang im Wizard, im Vergleich zu 10 Tests auf einer Seite, ca. 20 mal soviel Zeit ben�tigen.

Die Testzeiten erh�hen sich also in hohem Ma�e wenn mehrere Seiten getestet werden. Haupts�chlich wird dies durch die Latenz zum Server verursacht. Bei Tests �ber zwei Seiten wird zus�tzlich jedes mal der Wizard abgeschlossen, ein neues Sizing erstellt und wieder auf die zu testende Seite navigiert. Dieser Vorgang veranschlagt zwischen 5 und 20 Sekunden. Diese Spanne liegt so weit auseinander, da die Zeit die ben�tigt wird um wieder zur Testseite zu kommen sich mit jedem Testcase um ein bis vier Sekunden erh�ht.

\subsection{Evaluation der Testzeiten}

\section{Effektivit�t der Methode}
